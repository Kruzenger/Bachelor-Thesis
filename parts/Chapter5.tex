\section{Заключение}
\label{sec:Chapter5} \index{Chapter5}
    Настоящее исследование посвящено проектированию, разработке и реализации прототипа системы потоковой обработки данных социальных сетей, ориентированной на анализ пользовательского контента в режиме реального времени. В ходе работы достигнуты следующие ключевые результаты:

    \begin{enumerate}
        \item Предложена \textbf{гибридная архитектура}, интегрирующая принципы \textbf{Kappa-архи-тектуры} (единый поток событий, репроцессинг данных) с \textbf{микросервисным подходом}. Это позволило ликвидировать дублирование логики обработки и обеспечить модульность и независимое масштабирование компонентов.
        
        \item Разработан \textbf{механизм сбора данных} VK на базе асинхронного фреймворка \textbf{vkbottle}, обеспечивающий:
        \begin{itemize}
            \item Обход ограничений rate limit (3 запроса/секунду) через динамическое распределение задач по API-токенам
            \item Минимизацию задержек за счёт оптимизированной очереди задач
            \item Гибкое управление источниками через Fetcher Admin Service
        \end{itemize}
        
        \item Реализован \textbf{прототип конвейера обработки} на базе \textbf{Apache Kafka} (шина событий с "ровно один раз" семантикой) удовлетворяющий требованию задержки < 60 секунд. Исследована возможность подключения к конвееру Apache Flink для потоковой обработки и анализа данных
        
        Исследована и разработана архитектура, учитывающая \textbf{двухуровневую систему хранения}:
        \begin{itemize}
            \item \textbf{Elasticsearch} для оперативного поиска и визуализации
            \item \textbf{Cassandra} для долгосрочного хранения и интеграции с пакетной аналитикой
        \end{itemize}
    \end{enumerate}

    \subsection{Практическая значимость} 
        Практическая значимость работы заключается в создании универсальной платформы для маркетингового анализа (тренды, репутация брендов), социологических исследований (динамика настроений) и кризисного мониторинга (детекция аномалий)

    \subsection{Перспективные направления развития}:
        Работа имеет перспективные направления для развития в области Обработки данных. В качестве направлений развития выделены следующие варианты:
        \begin{itemize}
            \item Добавление поддержка Telegram/Twitter через модули Fetcher Admin Service
            \item Полноценная интеграция Apache Flink для потоковой обработки детализированных
            \item Реализация хранилищ данных на основе исследованных вариантов.
            \item Исследование использования шардирования в Cassandra
            \item Внедрение ML-пайплайнов для классификации контента
            \item Оптимизация Flink-операторов для снижения задержки до 10-20 секунд
            \item Исследование методов сжатия данных в Kafka/Cassandra
        \end{itemize}

    Таким образом, разработанное решение подтверждает эффективность гибридного подхода для задач потоковой обработки Big Data, сочетая оперативность Kappa-архитектуры с гибкостью микросервисов, и открывает пути для дальнейшей оптимизации в условиях экспоненциального роста социальных данных.

\newpage
