\section{Введение}
\label{sec:Chapter0} \index{Chapter0}
    Начало текущего тысячелетия отличилось стремительным развитием цифровых технологий, которые проникли почти во все сферы человеческой деятельности. Одним из ключевых драйверов этого процесса стало повсеместное использование социальных сетей, генерирующих колоссальные объёмы данных в режиме реального времени. Эти данные содержат ценную информацию о поведении пользователей, их предпочтениях и социальных взаимодействиях, что делает их критически важными для бизнеса, маркетинга, социологических исследований и управления общественными процессами. Однако обработка таких данных сопряжена с рядом вызовов, включая необходимость обеспечения высокой производительности, масштабируемости и низкой задержки при анализе потоковой информации.

    \medskip
    Существующие решения для работы с большими данными, такие как пакетная обработка, демонстрируют ограниченную эффективность в контексте динамичных социальных сетей, где задержки в анализе могут привести к утрате актуальности информации. В этой связи особую значимость приобретают системы потоковой обработки, способные агрегировать и анализировать данные в реальном времени. Подобные системы позволяют выявлять тенденции, обнаруживать аномалии и принимать оперативные решения на основе актуальных данных. 

    \subsection{Определения используемые в данной работе}
        Введём несколько определений используемых в данной работе.
        \begin{itemize}
            \item Потоковая обработка данных — это процесс обработки и объединения данных в режиме реального времени по мере их поступления из непрерывных источников.
            \item Пакетная обработка данных — это процесс обработки, где данные накапливаются и анализируются постфактум.
            \item Бэкенд (Backend) — это внутренняя часть программной или информационной системы, которая отвечает за ее функциональность и обработку данных. 
            \item Фронтенд (Frontend) в контексте веб-разработки относится к той части веб-прило-жения, которую видит пользователь и с которой он непосредственно взаимодействует
            \item СУБД — это программа, которая позволяет создавать базы данных и управлять ими. Она обеспечивает надёжное хранение информации, а также её безопасность и целостность. Помимо этого, СУБД предоставляет администратору средства для управления базой данных.
            \item API (Application Programming Interface) — это программный интерфейс, который позволяет одной компьютерной программе взаимодействовать с другими программами. Он предоставляет набор функций, объектов или действий, которые могут быть использованы разработчиками для создания приложений.
            \item IoT Мониторинг (IoT-monitoring, от Internet of Things) — процесс отслеживания, анализа и управления производительностью и здоровьем подключенных устройств в экосистеме Internet of Things. Он обеспечивает бесперебойную работу систем, повышает их эффективность и безопасность, предоставляя данные в режиме реального времени для дальнейшего анализа и упреждения проблем.
            \item JSON (расшифровывается как JavaScript Object Notation) — это облегченный формат для хранения и передачи данных. Он использует человекочитаемый текст для представления объектов данных, состоящих из пар <ИМЯ\_СВОЙСТВА>: <ЗНАЧЕНИЕ> и массивов, что упрощает его понимание и использование в различных языках программирования.
            \item RPC (Remote Procedure Call) —
            \item RDD (Resilient Distributed Datasets)
        \end{itemize}

    \subsection{Актуальность темы}
        В современных условиях социальные сети играют центральную роль в формировании информационной среды. Объёмы и скорость публикации данных в соцмедиа (Twitter, Facebook, VK и др.) растут крайне быстро: пользователи ежесекундно публикуют миллионы публикаций, фотографий и видео. Эти данные крайне ценны для самых разных областей — от маркетинга (анализ потребительских предпочтений и отслеживание репутации брендов) до общественной безопасности (мониторинг чрезвычайных ситуаций, выявление фейковых новостей или экстремистских проявлений). Однако эффективная работа с такими потоками требует специальных архитектур и технологий, способных обрабатывать данные в реальном времени с высокими объёмами и низкими задержками. Дело в том, что информация крайне быстро теряет актуальность, задержка в несколько минут или часов может сделать информацию устаревшей. В условиях постоянного роста объёмов данных традиционные методы пакетной обработки уже не отвечают требованиям времени. Отслеживание в потоковом режиме трендов и тенденций, ключевых событий или настроений аудитории даёт существенное преимущество в принятии решений в сфере маркетинга, в то время как своевременное обнаружение дезинформации может оказать значительное влияниие на общественную информацию. Поэтому возникает необходимость перехода к архитектурам, способным обеспечивать масштабируемость, отказоустойчивость и низкие задержки.
\newpage
