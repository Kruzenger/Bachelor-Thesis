\begin{abstract}

    \begin{center}
        \large{Проектирование системы для потоковой обработки и агрегации данных социальных сетей} \\
    \large\textit{Кидун Станислав Русланович} \\[1 cm]

    Данное исследование направлено на разработку высокопроизводительной системы потоковой обработки данных социальных сетей в реальном времени. Предложена гибридная архитектура, интегрирующая принципы Каппа-архитектуры для устранения дублирования логики и микросервисный подход для обеспечения модульности и горизонтальной масштабируемости. Реализовано решение на базе Apache Kafka (шина событий с "ровно один раз" семантикой) и Apache Flink (движок потоковой обработки), гарантирующее задержку менее 60 секунд от публикации поста в VK до его обработки. Система протестирвоана на основе данных VK. Разработан механизм сбора данных через асинхронный фреймворк vkbottle с динамическим распределением нагрузки по API-токенам, обходящий ограничение на частоту запросов (3 запроса/секунду). Система рассматривает возможности реализации системы хранения данных: Elasticsearch для оперативной аналитики и полнотекстового поиска, Cassandra — для долгосрочного архивирования и интеграции с пакетной обработкой. Практическая проверка подтвердила линейную масштабируемость: обработка 500 сообществ требует 3 токена VK API, 10 000 сообществ — 50 токенов при сохранении целевой задержки. Разработанная платформа имеет практическую значимость для маркетингового анализа трендов, социологических исследований и кризисного мониторинга аномалий, демонстрируя эффективность гибридного подхода для задач Big Data и открывая пути для интеграции новых источников (Telegram, Twitter) и ML-моделей классификации контента.
    \end{center}

\end{abstract}
\newpage