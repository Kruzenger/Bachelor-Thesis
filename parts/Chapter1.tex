\section{Постановка задачи}
\label{sec:Chapter1} \index{Chapter1}
    \subsection{Цели работы}
        Цель этого исследования заключается в исследовании методов обработки и агрегации данных, разработке собственной системы, а так же поиска возможностей для оптимизации и улучшения производительности в парадигме работы с данными социальных сетей. 
        В силу больших объёмов данных система должна быть способной обеспечить высокую производительность а так же обладать качествами высоконагруженных систем, таких как отказоустойчивость, высокую скорость отклика и быстрое восстановление после падения. Так же в силу разнообразности соцмедиа и отсутствия единых стандартов взаимодействия с ними, система должна обладать высокой гибкостью, что бы обеспечить возможность другим разработчикам создавать свои собственные модули без необхадимости перерабатывать уже разработанные модули. Так же одним из самых важных аспектов системы должно быть оперативное принятие решений на основе анализа поступающих данных. Итого, финальная модель должна соответствовать следующим критериям:
        \begin{itemize}
            \item \textbf{Масштабируемость.} Количество пользователей соцсетей продолжает расти, и система должна устойчиво работать при увеличении трафика в десятки раз без радикальной переработки инфраструктуры.
            \item \textbf{Отказоустойчивость.} Система должна уметь быстро восстанавливаться после падения, для скорейшего восстановления обработки трафика. 
            \item \textbf{Гибкость разработки. Модульность.} В связи с большим разнообразием социальных сетей и возможным появлением новый в будущем, система должна иметь возможность интеграции модулей работы с ними без кординальной переработки.
            \item \textbf{Гибридность с офлайновой аналитикой.} Компании и исследовательские организации хотят сочетать быструю первичную аналитику с более глубокой офлайновой аналитикой на больших исторических объёмах данных. Правильное проектирование стриминговой системы должно учитывать интеграцию с хранилищами больших данных и системами batch-аналитики.
            \item \textbf{Низкая задержка и высокая скорость отклика.} В рамках социальных сетей данные могут быстро терять актуальность поэтому система должна быстро реагировать на новые данные.
        \end{itemize}

    \subsection{Задачи работы}
        Для достижения поставленной цели были сформулированы следующие задачи:
        \begin{itemize}
            \item Изучить и проанализировать современные технологии и архитектуры потоковой обработки данных. Исследовать возможность их применения для работы с социальными сетями.
            \item Определить основные функциональные и нефункциональные требования к системе с учётом особенностей источников данных и существующих компонентов и решений.
            \item Разработать архитектурное решение, включающее выбор подходящих компонентов для обработки, агрегации и хранения данных.
            \item Реализовать прототип системы на основе выбранных технологий и провести его экспериментальную проверку.
            \item Проанализировать результаты экспериментов и оценить эффективность разработанного решения в соответствии с поставленными требованиями и критериями.
            \item Провести анализ написанной системы на возможности её улучшения и расширения в будущем.
        \end{itemize}

\newpage
